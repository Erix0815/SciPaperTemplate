\documentclass[a4paper,12pt]{article}
\usepackage[left=2.5cm,right=2.5cm,top=2.5cm,bottom=2.5cm,includehead]{geometry}      % Einstellungen der Seitenränder
\usepackage[english, ngerman]{babel}                                                  % deutsche Silbentrennung
\usepackage[utf8]{inputenc}                                                           % Umlaute
\usepackage[official]{eurosym}                                                        % Euro Symbol
\usepackage[T1]{fontenc}													                                    % Umlaute auch richtig ausgeben
\usepackage{newtxtext,newtxmath}                                                      % Font = Times New Roman
\usepackage{hyperref}
\usepackage[nottoc]{tocbibind}
\usepackage{fancyhdr}
\usepackage{setspace}
\usepackage[backend=bibtex, sorting=none, citestyle=authoryear, bibstyle=authoryear]{biblatex} % Bibliothek für Zitate
\usepackage{csquotes}                                                                 % Zusatzpacket für Zitate
\usepackage{amsmath}                                                                  % Zurücksetzen der Tabellen- und Abbildungsnummerierung je Sektion
\usepackage[labelfont=bf,aboveskip=1mm]{caption}                                      % Bild- und Tabellenunterschrift (fett)
\usepackage[bottom,multiple,hang,marginal]{footmisc}                                  % Fußnoten [Ausrichtung unten, Trennung durch Seperator bei mehreren Fußnoten]
\usepackage{graphicx}
\graphicspath{{./images/}}                                                            % Grafiken
\usepackage[dvipsnames]{xcolor}                                                       % Farbige Buchstaben
\usepackage{wrapfig}                                                                  % Bilder in Text integrieren
\usepackage{enumitem}                                                                 % Befehl setlist (Zeilenabstand für itemize Umgebung auf 1 setzen)
\usepackage{listings}                                                                 % Quelltexte
\definecolor{commentgreen}{RGB}{87,166,74}                                            % Kommentar-Farbe für Quellcode
\lstset{numbers=left, numberstyle=\tiny, numbersep=8pt, frame=single, framexleftmargin=15pt, breaklines=true, commentstyle=\color{commentgreen}}
\usepackage{tabularx}                                                                 % Tabellen
\usepackage{multirow}                                                                 % Mehrzeilige Tabelleneinträge
\usepackage{multicol}
\usepackage[addtotoc]{abstract}                                                       % Abstract
\usepackage[printonlyused, withpage]{acronym}                           % Abkürzungen
\usepackage{dirtree}                                                                  % Ordnerstruktur (z.B. für Anhang)
\usepackage[output=latex]{plantuml}
\usepackage{todonotes}
\newcommand{\td}[1]{\todo[inline, color=magenta, textcolor=blue, bordercolor=violet]{ToDo: #1}}

%%%%%%%%%%%%%%%%%%%%%%%%%%%%%%%%%%%%%%%%%%%%%%%%%%%%%%%%%%%%%%%%%%%%
%%%                      Angaben zur Arbeit                      %%%
%%%%%%%%%%%%%%%%%%%%%%%%%%%%%%%%%%%%%%%%%%%%%%%%%%%%%%%%%%%%%%%%%%%%
\def\vFirmenlogoPfad{images/fpt_logo.png}                  %% relativer Pfad Bsp.: images/Firmenlogo.png
\def\vDHBWLogoPfad{images/DHBW_logo.jpg}                          %% relativer Pfad Bsp.: images/DHBW_logo.jpg
\def\vUnterschrift{images/unterschrift.png}                    %% Pfad zu Bild mit Unterschrift (für digitale Abgabe) Bsp.: images/Unterschrift.png

\def\vTitel{}                           %% 
\def\vUntertitel{}                      %% 
\def\vArbeitstyp{}                      %% Projektarbeit/Seminararbeit/Bachelorarbeit
\def\vArbeitsbezeichnung{}              %% T1000/T2000/T3000

\def\vAutor{Eric Erath}                           %% Vorname Nachname
\def\vMatrikelnummer{1910897}                  %% 7-stellige Zahl
\def\vKursKuerzel{TIT22}                     %% Bsp.: TIT20
\def\vPhasenbezeichnung{}               %% Praxisphase/Theoriephase
\def\vStudienJahr{}                     %% erste/zweite/dritte
\def\vDHBWStandort{Ravensburg}                    %% Bsp.: Ravensburg
\def\vDHBWCampus{Friedrichshafen}                      %% Bsp.: Friedrichshafen
\def\vFakultaet{Technik}                       %% Technik/Wirtschaft
\def\vStudiengang{Informationstechnik}                     %% Informationstechnik/...

\def\vBetrieb{fpt Systems GmbH}                         %% 
\def\vBearbeitungsort{Hergensweiler}                 %% 
\def\vAbteilung{}                       %% 
\def\vBetreuer{}                        %% Vorname Nachname

\def\vAbgabedatum{\today}               %% DD. MONTH YYYY
\def\vBearbeitungszeitraum{01. 07. 2024 - 02. 08. 2024}            %% DD.MM.YYYY - DD.MM.YYYY


%%%%%%%%%%%%%%%%%%%%%%%%% Eigene Kommandos %%%%%%%%%%%%%%%%%%%%%%%%%
% Definition von \gqq{} und \gq{}: Text in Anführungszeichen
\newcommand{\gqq}[1]{\glqq #1\grqq}
\newcommand{\gq}[1]{\glq #1\grq}
% Spezielle Hervorhebung von Schlüsselwörtern
\newcommand{\fqq}[1]{\glqq \texttt{#1}\grqq}
\newcommand{\fq}[1]{\glq \texttt{#1}\grq}
\newcommand{\f}[1]{\texttt{#1}}


%%%%%%%%%%%%%%%%%%%% Zitatbibliothek einbinden %%%%%%%%%%%%%%%%%%%%%
\addbibresource{./literatur/literatur.bib}


%%%%%%%%%%%%%%%%%%%%%%%% PDF-Einstellungen %%%%%%%%%%%%%%%%%%%%%%%%%
\hypersetup{
  bookmarksopen=false,
	bookmarksnumbered=true,
	bookmarksopenlevel=0,
  pdftitle=\vTitel,
  pdfsubject=\vTitel,
  pdfauthor=\vAutor,
  pdfborder={0 0 0},
	pdfstartview=Fit,
  pdfpagelayout=SinglePage
}


%%%%%%%%%%%%%%%%%%%%%%%% Kopf- und Fußzeile %%%%%%%%%%%%%%%%%%%%%%%%
\pagestyle{fancy}
\setlength{\headheight}{15pt}
\fancyhf{}
\fancyhead[R]{\thepage}


%%%%%%%%%%%%%%%%%%%%%%%%%%%%%% Layout %%%%%%%%%%%%%%%%%%%%%%%%%%%%%%
\onehalfspacing
\setlist{noitemsep}

\addto\captionsngerman{
  \renewcommand{\figurename}{Abb.}
  \renewcommand{\tablename}{Tab.}
}
\numberwithin{table}{section}                               % Tabellennummerierung je Sektion zurücksetzen
\numberwithin{figure}{section}                              % Abbildungsnummerierung je Sektion zurücksetzen
\renewcommand{\thetable}{\arabic{section}.\arabic{table}}   % Tabellennummerierung mit Section
\renewcommand{\thefigure}{\arabic{section}.\arabic{figure}} % Abbildungsnummerierung mit Section
\renewcommand{\thefootnote}{\arabic{footnote}}              % Sektionsbezeichnung von Fußnoten entfernen

\renewcommand{\multfootsep}{, }                             % Mehrere Fußnoten durch ", " trennen


%%%%%%%%%%%%%%%%%%%%%%%%%%%%% Dokument %%%%%%%%%%%%%%%%%%%%%%%%%%%%%

\begin{document}
  %%%%%%%%%%%%%%%%%%% Einführung und Verzeichnisse %%%%%%%%%%%%%%%%%%%
  \pagenumbering{Roman}

  \include{pages/titel}
  \include{pages/sperrvermerk}
  \include{pages/selbststaendigkeitserklaerung}
  \include{pages/abstract}
  \section*{Vorwort}
\addcontentsline{toc}{section}{Vorwort}
In dieser Projektarbeit wird bewusst auf die Verwendung von geschlechterspezifischen Formulierungen verzichtet und das generische Maskulinum verwendet. Dies soll keinesfalls als gewollte Ausgrenzung von Gruppierungen gedeutet werden, sondern zur Lesbarkeit und Vermeidung von umständlichen Formulierungen beitragen. Weibliche und anderweitige Geschlechteridentitäten sind dabei ausdrücklich mitgemeint. Der Fokus dieser Arbeit liegt gänzlich auf der neutralen und wissenschaftlichen Behandlung des Themas, nicht auf der korrekten Einbindung aller Geschlechter.
\newpage

  %%%%%%%%%%%%%%%%%%%%%%%%%%%%%%%%
%% fuck whoever made multitoc %%
%%%%%%%%%%%%%%%%%%%%%%%%%%%%%%%%

\setcounter{tocdepth}{3}
\makeatletter
\def\mytoc{
    \section*{Inhaltsverzeichnis}
    \setlength{\columnseprule}{0.5pt}
    \setlength{\columnsep}{20pt}
    \begin{multicols*}{2}
        \@starttoc{toc}
        \listoftodos[\vfill]
    \end{multicols*}
}
\makeatother
\mytoc
\newpage

  \section*{Abkürzungsverzeichnis}
\addcontentsline{toc}{section}{Abkürzungsverzeichnis}
\begin{acronym}[Beckhoff] % update to be longest used acronym (+buffer)
  \acro{ads}[ADS]{\gqq{Automation Device Specification}}
  \acro{beckhoff}[Beckhoff]{\gqq{Beckhoff Automation GmbH & Co. KG}}
  \acro{blum}[Blum]{\gqq{Julius Blum GmbH}}
  \acro{bpmn}[BPMN]{\gqq{Business Process Modeling Notation}}
  \acro{cad}[CAD]{\gqq{Computer-aided Design}}
  \acro{dh}[DHBW]{\gqq{Duale Hochschule Baden-Württemberg}}
  \acro{fpt}[fpt]{\gqq{FPT Systems GmbH}}
  \acro{fup}[FUP]{\gqq{Functionblock Programming}}
  \acro{gft}[GFT]{\gqq{Gebindefördertechnik}}
  \acro{gml}[GML]{\gqq{Graphic Motion Language}}
  \acro{icon}[Icon-Prog]{\gqq{fpt ICON Programming}}
  \acro{krl}[KRL]{\gqq{Kuka Robot Language}}
  \acro{kuka}[Kuka]{\gqq{KUKA AG}}
  \acro{md}[MD]{\gqq{Markdown}}
  \acro{opc}[OPCUA]{\gqq{Open Platform Communication Unified Architecture}}
  \acro{siemens}[Siemens]{\gqq{Siemens AG}}
  \acro{sps}[SPS]{\gqq{Speicherprogrammierbare Steuerung}}
  \acro{sql}[SQL]{\gqq{Structured Query Language}}
  \acro{st}[ST]{\gqq{Structured Text}}
  \acro{stiwa}[Stiwa]{\gqq{STIWA Holding GmbH}}
  \acro{tcp}[TCP]{\gqq{Tool Center Point}}
  \acro{word}[Word]{\gqq{Microsoft Word}}
\end{acronym}
\newpage

  \include{pages/abbildungsverzeichnis}

  %%%%%%%%%%%%%%%%%%%%%%%%%%%%% Kapitel %%%%%%%%%%%%%%%%%%%%%%%%%%%%%%
  \pagestyle{fancy}
  \fancyhead[L]{\nouppercase{\rightmark}}    % Abschnittsname im Header
  \pagenumbering{arabic}

  %%%%%%%%%%%%%%%%%%%%%%%%%%%%%%%%%%%%%%%%%%%%%%%%%%%%%%%%%%%%%%%%%%%%
  %%%%                   EIGENE KAPITEL EINFÜGEN                  %%%%
  %%%%%%%%%%%%%%%%%%%%%%%%%%%%%%%%%%%%%%%%%%%%%%%%%%%%%%%%%%%%%%%%%%%%
  \section{Beispiele}
Hier befinden sich einige Beispiele zu tollen Sachen, die man in \TeX\LaTeX\TeX machen kann.

\subsection{Citation / Refs}
"Lorem Ipsum \ldots" \footcite[vgl.][S. 420]{Test} \\
Hier findest du tollen Code: Abb. \ref{CSharp} auf Seite \pageref{CSharp}

\subsection{PlantUML}
\begin{figure}[h]
  \centering
  \begin{plantuml}
    @startuml
    hide footbox
    !pragma teoz true

    participant Alice as a
    participant  Eve  as e
    participant  Bob  as b
    activate a
    activate e
    activate b

      a ->x e: Message
    & e x-> b: Message'

    @enduml
  \end{plantuml}
  \caption{Pumml}
  \label{puml}
\end{figure}

\subsection{Codeblocks}
\begin{figure}[h]
  \begin{lstlisting}[language={[Sharp]C}]
    #using System
    public static int main()
    {
      Console.Writeline("Hello world!");
      return 0;
    }
  \end{lstlisting}
  \caption{C\# - Code}
  \label{CSharp}
\end{figure}

\subsection{Tables}
\begin{figure}[h]
  \begin{tabularx}{\textwidth}{|r||X|c|}
    \hline & Spalte A & Spalte B \\
    \hline
    \hline Zeile 1 &
    Lorem ipsum dolor sit amet,
    consectetuer adipiscing elit.
    Etiam lobortis facilisis sem.
    Nullam nec mi et neque pharetra sollicitudin.
    & \LaTeX \\
    \hline Zeile 2 & He He! *moonwalks casually* & Micel Jakson \\
    \hline
  \end{tabularx}
  \caption{Tabbelllllle}
  \label{tab:tab}
\end{figure}

\newpage
\subsection{ToDos}
\td{Beispiel}
\listoftodos

  \include{chapter/Einleitung}
  \include{chapter/Hauptteil}
  \section{Reflexion}
\subsection{Zusammenfassung}
\subsection{Limitationen}
\section{Fazit}


  %%%%%%%%%%%%%%%%%%%%%%% Literaturverzeichnis %%%%%%%%%%%%%%%%%%%%%%%
  \include{pages/literaturverzeichnis}

  %%%%%%%%%%%%%%%%%%%%%%%%%%%%%% Anhang %%%%%%%%%%%%%%%%%%%%%%%%%%%%%%
  \renewcommand{\thetable}{\Alph{section}.\arabic{table}}
  \renewcommand{\thefigure}{\Alph{section}.\arabic{figure}}
  \renewcommand{\thelstlisting}{\Alph{section}.\arabic{lstlisting}}
  \pagenumbering{Alph}

  \begin{appendix}
  \section{Anhang}
  \DTsetlength{0.2em}{1em}{0.2em}{1pt}{2pt}
  \dirtree{%
    .1 Anhang.zip.
    .2 Quellen.
    .3 FA3278\_Blum\_IL43G\_Palettieren\_EPLAN.pdf.
    .3 FA3278\_Blum\_IL43G\_Palettieren\_GreiferM54\_Lastdaten\_V1.pdf.
    .3 Graphic\_Motion\_Language.lnk.
    .3 Graphic\_Motion\_Language.htm.
  }

  \bigskip
  \noindent
  Aus Geheimhaltungsgründen dürfen die Schulungsdokumente der KUKA Deutschland GmbH nicht als Quellennachweis angefügt werden.
\end{appendix}

\end{document}
